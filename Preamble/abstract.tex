\unnumberedchapter{Abstract} 
\chapter*{Abstract} 
%\subsection*{\thesistitle}


Birds are important indicators of the health of the environment. Birds pollinate plants, disperse plant seeds and suppress pest populations. Accurate bird recognition helps to quantify the effects of human activity on avian biodiversity. Most researchers use passive acoustic monitors (PAMs) to collect recordings spanning hundreds of hours resulting in large volumes of data. Due to the size of recordings, bird identification remains an almost impossible task to be done manually. As PAMs could be remotely located, the sound files may require substantial effort to collect.

The edge computing paradigm allows data processing tasks to be performed at the edge of the network. This new approach reduces system running time, memory requirements and energy consumption for a wide range of big data applications. Combined with long-range low-power radio networks, applying the approach for bird recognition should reduce the need to manually retrieve sound recordings which contain mostly redundant information anyway.

Bird sound recognition applies many of the same techniques as sound event detection and keyword spotting (KWS). For large scale bird recognition, the highest accuracies are currently achieved using ImageNet-class convolutional neural networks (CNNs). These large CNNs are computationally complex in terms of multiply-and-add operations and require a lot of memory. For edge computing, as an alternative is required to preserve battery life and reduce systems costs.

This research aims to identify and propose suitable low-complexity algorithms for bird sound detection and classification. The approach to apply the two-stage cascade architecture used in KWS \cite{Sigtia2018}. The first stage is the bird detector. It is a binary classifier that detects a bird sound of any kind. In a battery-powered device, the first stage is always turned on, therefore, performs minimal computations to reduce energy usage. The detector stage is to be constructed using a binarized convolutional network (BNN). The second stage is the bird classifier. The output of the classifier is the identified bird species. This stage requires more computational effort as it consists of an integer CNN. In a battery-power device, the second stage is turned on only when the first stage detects a bird sound to reduce energy consumption and improve recognition accuracy. In addition to exploring several BNN and CNN configuration, the research will investigate the impact of various audio features on recognition accuracy.

Preliminary works have been to collect sound files for 10 species of birds and to test the recognition accuracy on the ResNet-50 CNN using short-time Fourier transform (STFT) and Mel Frequency Cepstrum Coefficients (MFCC) features. %For the remainder of the research, 10 more species of birds will be added.
 Alternative CNN configurations and audio features will be explored. The accuracy and complexity of the proposed algorithms will be benchmarked against state-of-the-art results.